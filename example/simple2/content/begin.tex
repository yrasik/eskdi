\section*{Введение}
\sectionmark{Введение}

В  курсе  лекций  по  специальным  дисциплинам  таким  как радиотехнические цепи и сигналы, устройства формирования электрических сигналов,  устройства  приёма  и  обработки  сигналов  и  др.  линейные электрические  цепи а также цепи, приводимые к линейным анализируются методом чеырёхполюсников.  

Как известно, во многих случаях задача анализа рабочих режимов в электрических  цепях  ограничивается  расчётом  токов  и  напряжений  в отдельных участках цепей или нахождением уравнений связи между этими токами  и  напряжениями.  При  этом  режимы  остальных  участков  цепей остаются неизвестными,  хотя все их параметры учитываются при решении 
соответствующих  задач.  Тогда  рассматриваемую  часть  цепи  можно характеризовать  обобщёнными  параметрами  по  отношению  к  некоторым выделенным зажимам.  

Часть  цепи,  характеризуемую  обобщёнными  параметрами, необходимыми  и  достаточными  для  составления  уравнений  связи  между токами и напряжениями на её зажимах, называют многополюсниками. 

Реальная  схема  соединений  элементов  части  цепи,  составляющей многополюсник,  может  быть  неизвестна.  Число  плюсов  многополюсника равно  числу  зажимов  на  границе  данной  части  схемы.  При  исследовании режимов  в  электрических  цепях  чаще  всего  используют  двухполюсники, трёхполюсники и четырёхполюсники.  

С помощью метода четырёхполюсников возможно определение таких параметров  электрических  цепей  как  комплексный  коэффициент  передачи цепи по напряжению с учётом нагрузки на выходе и на входе цепи, комплексное  входное  сопротивление  с  учётом  нагрузки  на  выходе  а  также комплексное  выходное  сопротивление  с  учётом  нагрузки  на  входе. Результаты    моделирования  с  помощью  данного  метода  дают  практически полное  совпадение  с  данными,  полученными  с  помощью    программного продукта OrCAD~9.1. 

Нужно  отметить  что  хотя  с  помощью  пакта  моделирования  OrCAD~9.1. можно получить те же результаты,  но иногда из-за особенностей схемы это не всегда удобно. 

К недостаткам данного метода можно отнести сложность организации моделей индуктивностей с отводами. 

Ниже будут рассмотрены особенности применения данного метода в расчёте некоторых каскадов радиоприёмного устройства. 

