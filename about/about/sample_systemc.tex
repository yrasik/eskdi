\section{Инструкция по установке окружения SystemC + SystemC-AMS + TUV\_AMS\_Library под Win7}
\sectionmark{Инструкция по установке SystemC под Win7}

\subsection{О SystemC}

SystemC — язык проектирования и верификации моделей системного уровня, реализованный в виде C++ библиотеки с открытым исходным кодом. Библиотека включает в себя ядро событийного моделирования, что позволяет получить исполняемую модель устройства.

SystemC~AMS — дополнительная библиотека, реализованная на C++ с открытым исходным кодом, позволяющая моделировать аналоговые и гетерогенные системы.

TUV\_AMS\_Library — набор библиотечных модулей обработки сигналов, выполненный в виде классов SystemC~AMS.

SystemC + SystemC-AMS + TUV\_AMS\_Library представляют собой бесплатную альтернативу Matlab Simulink Toolbox.


\subsection{Необходимые пакеты}

\subsubsection{Библиотеки SystemC}

systemc-2.3.1.tgz  - ядро SystemC.

systemc-ams-2\_beta1.zip - рашширение, позволяющее моделировать аналоговые системы.

TUV\_AMS\_Library.tar.gz - библиотека с готовыми и отлаженными функциональными модулями.


\subsubsection{Интегрированная среда ''Microsoft Visual C++''}

В документации на библиотеки фигурирует ''Microsoft Visual C++ 2008'' и выше...


Сборка библиотек осуществлялась ''Microsoft Visual C++ 2008 Express''.

\subsubsection{Утилита для сборки ''cmake''}


cmake-3.3.0-rc2-win32-x86.exe - нужна для создания файлов проекта "Microsoft Visual C++" для "TUV\_AMS\_Library".






\subsection{Руководство по установке}

\subsubsection{Установка ''Microsoft Visual C++ 2008 Express''}
  Устанавливается согласно приложенной инструкции.
  
  
\subsubsection{Сборка библиотеки ''SystemC.lib''}

Запускается ''Microsoft Visual C++ 2008 Express Edition.exe''.

Распаковывается архив ''systemc-2.3.1.tgz'':
\begin{lstlisting}[language=, style=FrameStyleI]
d:\systemc\systemc-2.3.1
\end{lstlisting}


Открывается из ''Visual C++'' проект:
\begin{lstlisting}[language=, style=FrameStyleI]
d:\systemc\systemc-2.3.1\msvc80\SystemC.sln 
\end{lstlisting}

Запускается сборка: Build -> Build SystemC.

Получается библиотека в:

\begin{lstlisting}[language=, style=FrameStyleI]
d:\systemc\systemc-2.3.1\msvc80\SystemC\Debug\SystemC.lib
\end{lstlisting}





\subsubsection{Сборка библиотеки ''systemc\_ams.lib''}


Распаковывается архив ''systemc-ams-2\_beta1.zip'':
\begin{lstlisting}[language=, style=FrameStyleI]
d:\systemc\systemc-ams-2.0_beta1
\end{lstlisting}




Открывается из ''Visual C++'' проект:
\begin{lstlisting}[language=, style=FrameStyleI]
d:\systemc\systemc-ams-2.0_beta1\msvc80\systemc_ams.sln
\end{lstlisting}

Заголовочныые файлы от ''SystemC.lib'':

\begin{lstlisting}[language=, style=FrameStyleI]
Progect -> Properties ->  C/C++ -> General-> Additional Include Directories:
\end{lstlisting}

добавляем:

\begin{lstlisting}[language=, style=FrameStyleI]
"d:\systemc\systemc-2.3.1\include"
\end{lstlisting}

Путь к библиотеке ''SystemC.lib'':

\begin{lstlisting}[language=, style=FrameStyleI]
Progect -> Properties ->  Librarian -> General-> Additional Library Directories
\end{lstlisting}

добавляем:

\begin{lstlisting}[language=, style=FrameStyleI]
"d:\systemc\systemc-2.3.1\msvc80\SystemC\Debug"
\end{lstlisting}




Запускаем сборку:

\begin{lstlisting}[language=, style=FrameStyleI]
Build -> Build system_ams
\end{lstlisting}


Получаем библиотеку в:

\begin{lstlisting}[language=, style=FrameStyleI]
"d:\systemc\systemc-ams-2.0_beta1\msvc80\Debug\systemc_ams.lib"
\end{lstlisting}



\subsubsection{Сборка библиотеки ''tuv\_ams\_libraryd.lib''}

Запустить ''Visual Studio 2008 Command Prompt.exe''

Зайти в директорию:
\begin{lstlisting}[language=, style=FrameStyleI]
"d:\systemc\TUV_AMS_Library\build"
\end{lstlisting}



Прописать переменные окружения в консоли:

\begin{lstlisting}[language=, style=FrameStyleI]
set SYSTEMC_HOME=d:\systemc_2015\systemc-2.3.1\
\end{lstlisting}

\begin{lstlisting}[language=, style=FrameStyleI]
set SYSTEMCAMS_HOME=d:\systemc_2015\systemc-ams-2.0_beta1\
\end{lstlisting}





Скопировать библиотеки :

\begin{lstlisting}[language=, style=FrameStyleI]
d:\systemc\systemc-2.3.1\msvc80\SystemC\Debug\SystemC.lib
->
d:\systemc\systemc-2.3.1\SystemC.lib
\end{lstlisting}

\begin{lstlisting}[language=, style=FrameStyleI]
d:\systemc\systemc-ams-2.0_beta1\msvc80\Debug\systemc_ams.lib
->
d:\systemc\systemc-ams-2.0_beta1\systemc_ams.lib
\end{lstlisting}



Вбить в консоли команду

\begin{lstlisting}[language=, style=FrameStyleI]
cmake ..
\end{lstlisting}

Побежит лог конфигуратора. Сообщений об ошибках в логе не должно быть.




Открываем проект в ''Visual C++'' по адресу:

\begin{lstlisting}[language=, style=FrameStyleI]
"d:\systemc\TUV_AMS_Library\build\TUV_AMS.sln"
\end{lstlisting}

Запускаем сборку:


\begin{lstlisting}[language=, style=FrameStyleI]
Build -> Build system_ams
\end{lstlisting}


Получаем библиотеку в

\begin{lstlisting}[language=, style=FrameStyleI]
"d:\systemc\TUV_AMS_Library\build\Debug\tuv_ams_libraryd.lib"
\end{lstlisting}




\subsubsection{Сборка тестового примера ''Frequency\_shifter''}


Чтобы собрался проект ''Frequency\_shifter'' нужно:

закомментировать строчку :

\begin{lstlisting}[language=C++, style=FrameStyleI]
ReadWaveFile.out.set_timestep(0.01, SC_US);
\end{lstlisting}        

вписать метод в:
     
\begin{lstlisting}[language=C++, style=FrameStyleI]
SCA_TDF_MODULE(Input)
{
  ...
  void set_attributes()
  {
    out.set_timestep(0.01, SC_US);
  }
  ...
}
\end{lstlisting}     
     
     





Пример собирается, но вылетает с ошибкой переполнение стэка с параметром N = 256 (размерность БПФ преобразований). При N = 64 пример работает.  Оно и после сборки под cygwin ведёт себя подобным образом.



!!! УВЕЛИЧИТЬ СТЭК ПРИЛОЖЕНИЯ (т.к. Frequency\_shifter вылетает с настройками по умолчанию):

\begin{lstlisting}[language=, style=FrameStyleI]
MVS -> Properties -> Linker -> System -> Stack Reserve Size = 100000000
&
MVS -> Properties -> Linker -> System -> Stack Commit Size = 100000000
\end{lstlisting}












