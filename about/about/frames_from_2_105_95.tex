\section{Выдержки из стандарта ''ГОСТ~2.105–95''}
\sectionmark{Выдержки из стандарта ''ГОСТ~2.105–95''}

\subsectionImit{3 ОБЩИЕ ПОЛОЖЕНИЯ}


3.2 Текстовые документы выполняют на формах, установленных соответствующими стандартами Единой системы конструкторской документации (ЕСКД) и Системы проектной документации для строительства (СПДС).

Требования, специфические для некоторых видов текстовых документов (например эксплуатационных документов), приведены в соответствующих стандартах.

3.5 Вписывать в текстовые документы, изготовленные машинописным способом, отдельные слова,
формулы, условные знаки (рукописным способом), а также выполнять иллюстрации следует чёрными
чернилами, пастой или тушью.

3.6 Расстояние от рамки формы до границ текста в начале и в конце строк – не менее 3~мм.
Расстояние от верхней или нижней строки текста до верхней или нижней рамки должно быть не менее 10 мм.

Абзацы в тексте начинают отступом, равным пяти ударам пишущей машинки (15 – 17~мм).

3.8 Для размещения утверждающих и согласующих подписей к текстовым документам рекомендуется
составлять титульный лист и (или) лист утверждения в соответствии с разделом~6 настоящего стандарта.

Обязательность и особенности выполнения титульных листов оговорены в стандартах ЕСКД и СПДС на правила выполнения соответствующих документов.

3.9 К текстовым документам рекомендуется выпускать лист регистрации изменений в соответствии с ГОСТ~2.503 и ГОСТ~21.101.

\subsectionImit{4 ТРЕБОВАНИЯ К ТЕКСТОВЫМ ДОКУМЕНТАМ, СОДЕРЖАЩИМ,
В ОСНОВНОМ, СПЛОШНОЙ ТЕКСТ}

4.1 Построение документа

4.1.1 Текст документа при необходимости разделяют на разделы и подразделы.

При большом объёме документа допускается разделять его на части, а части, в случае необходимости, на книги. Каждую часть и книгу комплектуют отдельно. Всем частям дают наименования и присваивают обозначение документа. Начиная со второй части, к этому обозначению добавляют порядковый номер,
например:

ХХХХ.331112.032Ф0, ХХХХ.331112.032Ф01, ХХХХ.331112.032Ф02, и т.д. Всем книгам дают наименование
и присваивают порядковый номер. Пример заполнения поля 4 титульного листа на книгу приведён в приложении~Б.
Листы документа нумеруют в пределах каждой части, каждую часть начинают на листах с основной надписью по форме ГОСТ~2.104 и форме~3 ГОСТ~Р~21.1101.

4.1.2 Разделы должны иметь порядковые номера в пределах всего документа (части, книги), обозначенные арабскими цифрами без точки и записанные с абзацевого отступа. Подразделы должны иметь нумерацию в пределах каждого раздела. Номер подраздела состоит из номеров раздела и подраздела, разделённых точкой. В конце номера подраздела точка не ставится. Разделы, как и подразделы, могут состоять из одного или
нескольких пунктов.

4.1.3 Если документ не имеет подразделов, то нумерация пунктов в нем должна быть в пределах каждого раздела, и номер пункта должен состоять из номеров раздела и пункта, разделённых точкой. В конце номера пункта точка не ставится.

Если документ имеет подразделы, то нумерация пунктов должна быть в пределах подраздела и номер пункта должен состоять из номеров раздела, подраздела и пункта, разделённых точками.

4.1.4 Если раздел или подраздел состоит из одного пункта, он также нумеруется.

4.1.5 Если текст документа подразделяется только на пункты, они нумеруются порядковыми номерами в пределах документа.

4.1.6 Пункты, при необходимости, могут быть разбиты на подпункты, которые должны иметь порядковую нумерацию в пределах каждого пункта, например: 4.2.1.1, 4.2.1.2, 4.2.1.3 и т.д.

4.1.7 Внутри пунктов или подпунктов могут быть приведены перечисления.
Перед каждой позицией перечисления следует ставить дефис или при необходимости ссылки в тексте документа на одно из перечислений, строчную букву, после которой ставится скобка. Для дальнейшей детализации перечислений необходимо использовать арабские цифры, после которых ставится скобка, а запись производится с абзацного отступа.

4.1.8 Каждый пункт, подпункт и перечисление записывают с абзацного отступа.

4.1.9 Разделы, подразделы должны иметь заголовки. Пункты, как правило, заголовков не имеют.

Заголовки должны чётко и кратко отражать содержание разделов, подразделов.

Заголовки следует печатать с прописной буквы без точки в конце, не подчёркивая. Переносы слов в заголовках не допускаются. Если заголовок состоит из двух предложений, их разделяют точкой.

Расстояние между заголовком и текстом при выполнении документа машинописным способом должно быть равно 3,4 интервалам, при выполнении рукописным способом – 15~мм. Расстояние между заголовками раздела и подраздела – 2 интервала, при выполнении рукописным способом – 8~мм.

4.1.10 Каждый раздел текстового документа рекомендуется начинать с нового лист (страницы).

4.1.11 В документе (части, книге) большого объёма на первом (заглавном) листе и, при необходимости, на последующих листах помещают содержание, включающее номера и наименования разделов и подразделов с указанием номеров листов (страниц).

Если документ разбит на части (книги), то в конце содержания первой части (книги) перечисляют обозначение и наименование (при наличии) остальных частей (книг). Содержание включают в общее количество листов данного документа (части, книги).

Слово ''Содержание'' записывают в виде заголовка (симметрично тексту) с прописной буквы. Наименования, включённые в содержание, записывают строчными буквами, начиная с прописной буквы.

4.1.12 В конце текстового документа перед листом регистрации изменений допускается приводить список литературы, которая была использована при его составлении. Выполнение списка и ссылки на него в тексте - по ГОСТ~7.32. Список литературы включают в содержание документа.

4.1.13. Нумерация страниц документа и приложений, входящих в состав этого документа, должна быть сквозная. Допускается вместо сквозной нумерации страниц применять нумерацию страниц в пределах каждого раздела документа.

4.2 Изложение текста документов

4.2.1 Полное наименование изделия на титульном листе, в основной надписи и при первом упоминании в тексте документа должно быть одинаковым с наименованием его в основном конструкторском документе.

В последующем тексте порядок слов в наименовании должен быть прямой, т.е. на первом месте должно быть определение (имя прилагательное), а затем - название изделия (имя существительное); при этом допускается употреблять сокращённое наименование изделия.

Наименования, приводимые в тексте документа и на иллюстрациях, должны быть одинаковыми.

4.2.2 Текст документа должен быть кратким, чётким и не допускать различных толкований.

При изложении обязательных требований в тексте должны применяться слова ''должен'', ''следует'', ''необходимо'', ''требуется, чтобы'', ''разрешается только'', ''не допускается'', ''запрещается'', ''не следует''. При изложении других положений следует применять слова - ''могут быть'', ''как правило'', ''при необходимости'',
''может быть'', ''в случае'' и т.д.

При этом допускается использовать повествовательную форму изложения текста документа, например ''применяют'', ''указывают'' и т.п.

В документах должны применяться научно–технические термины, обозначения и определения, установленные соответствующими стандартами, а при их отсутствии – общепринятые в научно–технической литературе.

Если в документе принята специфическая терминология, то в конце его (перед списком литературы) должен быть перечень принятых терминов с соответствующими разъяснениями. Перечень включают в содержание документа.

4.2.3 В тексте документа не допускается:
\begin{itemize}
  \item применять обороты разговорной речи, техницизмы, профессионализмы;
  \item применять для одного и того же понятия различные научно–технические термины, близкие по смыслу (синонимы), а также иностранные слова и термины при наличии равнозначных слов и терминов в русском языке;
  \item применять произвольные словообразования;
  \item применять сокращения слов, кроме установленных правилами русской орфографии, соответствующими государственными стандартами, а также в данном документе;
  \item сокращать обозначения единиц физических величин, если они употребляются без цифр, за исключением единиц физических величин в головках и боковиках таблиц, и в расшифровках буквенных обозначений, входящих в формулы и рисунки.
\end{itemize}


4.2.4 В тексте документа, за исключением формул, таблиц и рисунков, не допускается:
\begin{itemize}
  \item применять математический знак минус (–) перед отрицательными значениями величин (следует писать слово ''минус'');
  \item применять знак ''\O'' для обозначения диаметра (следует писать слово ''диаметр''). При указании размера или предельных отклонений диаметра на чертежах, помещённых в тексте документа, перед размерным числом следует писать знак ''\O'';
  \item применять без числовых значений математические знаки, например $>$ (больше), $<$ (меньше), $=$ (равно), $\geq$ (больше или равно), $\leq$ (меньше или равно), $\neq$ (не равно), а также знаки \textnumero (номер), \% (процент);
  \item применять индексы стандартов, технических условий и других документов без регистрационного номера.
\end{itemize}

4.2.5 Если в документе приводятся поясняющие надписи, наносимые непосредственно на изготовляемое изделие (например на планки, таблички к элементам управления и т.п.), их выделяют шрифтом (без кавычек), например ВКЛ., OTKЛ., или кавычками – если надпись состоит из цифр и (или) знаков.

Наименования команд, режимов, сигналов и т.п. в тексте следует выделять кавычками, например, ''Сигнал +27 включено''.

4.2.6 Перечень допускаемых сокращений слов установлен в ГОСТ~2.316.

Если в документе принята особая система сокращения слов или наименований, то в нем должен быть приведён перечень принятых сокращений, который помещают в конце документа перед перечнем терминов.

4.2.7 Условные буквенные обозначения, изображения или знаки должны соответствовать принятым в действующем законодательстве и государственных стандартах. В тексте документа перед обозначением параметра дают его пояснение, например ''Временное сопротивление разрыву $\sigma_В$ ''.

При необходимости применения условных обозначений, изображений или знаков, не установленных действующими стандартами, их следует пояснять в тексте или в перечне обозначений.

4.2.8 В документе следует применять стандартизованные единицы физических величин, их наименования и обозначения в соответствии с ГОСТ~8.417.

Наряду с единицами СИ, при необходимости, в скобках указывают единицы ранее применявшихся систем, разрешённых к применению. Применение в одном документе разных систем обозначения физических величин не допускается.

4.2.9 В тексте документа числовые значения величин с обозначением единиц физических величин и единиц счета следует писать цифрами, а числа без обозначения единиц физических величин и единиц счета от единицы до девяти – словами.

4.2.10 Единица физической величины одного и того же параметра в пределах одного документа должна быть постоянной. Если в тексте приводится ряд числовых значений, выраженных в одной и той же единице физической величины, то ее указывают только после последнего числового значения, например 1,50; 1,75; 2,00~м.

4.2.11 Если в тексте документа приводят диапазон числовых значений физической величины, выраженных в одной и той же единице физической величины, то обозначение единицы физической величины указывается после последнего числового значения диапазона.

Недопустимо отделять единицу физической величины от числового значения (переносить их на разные строки или страницы), кроме единиц физических величин, помещаемых в таблицах, выполненных машинописным способом.

4.2.12 Приводя наибольшие или наименьшие значения величин, следует применять словосочетание ''должно быть не более (не менее)''.

Приводя допустимые значения отклонений от указанных норм, требований, следует применять словосочетание ''не должно быть более (менее)''.

Например, массовая доля углекислового натрия в технической кальцинированной соде должна быть не менее 99,4~\%.

4.2.13 Числовые значения величин в тексте следует указывать со степенью точности, которая необходима для обеспечения требуемых свойств изделия, при этом в ряду величин осуществляется выравнивание числа знаков после запятой.

Округление числовых значений величин до первого, второго, третьего и т.д. десятичного знака для различных типоразмеров, марок и т.п. изделий одного наименования должно быть одинаковым. Например, если градация толщины стальной горячекатаной ленты 0,25~мм, то весь ряд толщин ленты должен быть указан с таким же количеством десятичных знаков, например 1,50; 1,75; 2,00.

4.2.15 В формулах в качестве символов следует применять обозначения, установленные соответствующими государственными стандартами. Пояснения символов и числовых коэффициентов, входящих в формулу, если они не пояснены ранее в тексте, должны быть приведены непосредственно под формулой. Пояснения каждого символа следует давать с новой строки в той последовательности, в которой символы приведены в формуле.
Первая строка пояснения должна начинаться со слова ''где'' без двоеточия после него.

Пример – Плотность каждого образца $\rho$, кг/м\textthreesuperior , вычисляют по формуле

\begin{equation*}
\rho = \frac{m}{V},  \tag{1}
\end{equation*}
\par где  $m$~– масса образца, кг;
\par $V$~– объем образца, м\textthreesuperior.

Формулы, следующие одна за другой и не разделённые текстом, разделяют запятой.

4.2.16 Переносить формулы на следующую строку допускается только на знаках выполняемых операций, причём знак в начале следующей строки повторяют. При переносе формулы на знаке умножения применяют знак ''\texttimes''.

4.2.18 Формулы, за исключением формул, помещаемых в приложении, должны нумероваться сквозной нумерацией арабскими цифрами, которые записывают на уровне формулы справа в круглых скобках. Одну формулу обозначают – (1).

Ссылки в тексте на порядковые номера формул дают в скобках, например, ... в формуле (1).

Формулы, помещаемые в приложениях, должны нумероваться отдельной нумерацией арабскими цифрами в пределах каждого приложения с добавлением перед каждой цифрой обозначения приложения, например формула (В.1).

Допускается нумерация формул в пределах раздела. В этом случае номер формулы состоит из номера раздела и порядкового номера формулы, разделённых точкой, например (3.1).

4.2.19 Порядок изложения в документах математических уравнений такой же, как и формул.

4.2.20 Примечания приводят в документах, если необходимы пояснения или справочные данные к содержанию текста, таблиц или графического материала.

Примечания не должны содержать требований.

4.2.21 Примечания следует помещать непосредственно после текстового, графического материала или в таблице, к которым относятся эти примечания, и печатать с прописной буквы с абзаца. Если примечание одно, то после слова ''Примечание'' ставится тире и примечание печатается тоже с прописной буквы. Одно примечание не нумеруют. Несколько примечаний нумеруют по порядку арабскими цифрами. Примечание к таблице помещают в конце таблицы над линией, обозначающей окончание таблицы.

4.2.22 В текстовом документе допускаются ссылки на данный документ, стандарты, технические условия и другие документы при условии, что они полностью и однозначно определяют соответствующие требования и не вызывают затруднений в пользовании документом.

Ссылки на стандарты предприятий (СТП) и другую техническую документацию должны быть оговорены в договоре на разработку изделия.

Ссылаться следует на документ в целом или его разделы и приложения. Ссылки на подразделы, пункты, таблицы и иллюстрации не допускаются, за исключением подразделов, пунктов, таблиц и иллюстраций данного документа.

При ссылках на стандарты и технические условия указывают только их обозначение, при этом допускается не указывать год их утверждения при условии записи обозначения с годом утверждения в конце текстового документа под рубрикой ''ССЫЛОЧНЫЕ НОРМАТИВНЫЕ ДОКУМЕНТЫ'' по форме:

\begin{longtable}{|p{60mm}|p{100mm}|}
  \hline
  \multicolumn{1}{|p{60mm}|}{\centering Обозначение документа, на который дана
ссылка} &
  \multicolumn{1}{p{100mm}|}{\centering Номер раздела, подраздела, пункта, подпункта, перечисления, приложения, разрабатываемого документа, в котором дана ссылка} \\\hline
  \endfirsthead
 \hline
  \multicolumn{1}{|p{60mm}|}{\centering Обозначение документа, на который дана
ссылка} &
  \multicolumn{1}{p{100mm}|}{\centering Номер раздела, подраздела, пункта, подпункта, перечисления, приложения, разрабатываемого документа, в котором дана ссылка} \\\hline
  \endhead
       &      \\ 
\end{longtable}

При ссылках на другие документы в графе ''Обозначение документа'' указывают также и наименование документа. При ссылках на раздел или приложение указывают его номер.

4.3 Оформление иллюстраций и приложений

4.3.1 Количество иллюстраций должно быть достаточным для пояснения излагаемого текста. Иллюстрации могут быть расположены как по тексту документа (возможно ближе к соответствующим частям текста), так и в конце его. Иллюстрации должны быть выполнены в соответствии с требованиями стандартов ЕСКД и СПДС.
Иллюстрации, за исключением иллюстраций приложений, следует нумеровать арабскими цифрами сквозной нумерацией. Если рисунок один, то он обозначается ''Рисунок 1''.

Иллюстрации каждого приложения обозначают отдельной нумерацией арабскими цифрами с добавлением перед цифрой обозначения приложения. Например – Рисунок~А.3.

Допускается нумеровать иллюстрации в пределах раздела. В этом случае номер иллюстрации состоит из номера раздела и порядкового номера иллюстрации, разделённых точкой. Например – Рисунок 1.1.

При ссылках на иллюстрации следует писать ''... в соответствии с рисунком 2'' при сквозной нумерации и ''... в соответствии с рисунком 1.2'' при нумерации в пределах раздела.

Иллюстрации, при необходимости, могут иметь наименование и пояснительные данные (подрисуночный текст). Слово ''Рисунок'' и наименование помещают после пояснительных данных и располагают следующим образом: Рисунок 1 – Детали прибора.

4.3.3 На приводимых в документе электрических схемах около каждого элемента указывают его позиционное обозначение, установленное соответствующими стандартами, и при, необходимости, номинальное значение величины.

4.3.4 Материал, дополняющий текст документа, допускается помещать в приложениях. Приложениями могут быть, например, графический материал, таблицы большого формата, расчёты, описания аппаратуры и приборов, описания алгоритмов и программ задач, решаемых на ЭВМ и т.д.

Приложение оформляют как продолжение данного документа на последующих его листах или выпускают в виде самостоятельного документа.

4.3.5 Приложения могут быть обязательными и информационными.

Информационные приложения могут быть рекомендуемого или справочного характера.

4.3.6 В тексте документа на все приложения должны быть даны ссылки. Степень обязательности приложений при ссылках не указывается. Приложения располагают в порядке ссылок на них в тексте документа, за исключением информационного приложения ''Библиография'', которое располагают последним.

4.3.7 Каждое приложение следует начинать с новой страницы с указанием наверху посередине страницы слова ''Приложение'' и его обозначения, а под ним в скобках для обязательного приложения пишут слово ''обязательное'', а для информационного – ''рекомендуемое'' или ''справочное''.

Приложение должно иметь заголовок, который записывают симметрично относительно текста с прописной буквы отдельной строкой.

4.3.8 Приложения обозначают заглавными буквами русского алфавита, начиная с А, за исключением букв Ё, 3, Й, О, Ч, Ь, Ы, Ъ. После слова ''Приложение'' следует буква, обозначающая его последовательность.

Допускается обозначение приложений буквами латинского алфавита, за исключением букв I и О.

В случае полного использования букв русского и латинского алфавитов допускается обозначать приложения арабскими цифрами.

Если в документе одно приложение, оно обозначается ''Приложение А''.

4.3.9 Приложения, как правило, выполняют на листах формата А4. Допускается оформлять приложения на листах формата A3, А4\texttimes3, А4\texttimes4, А2 и А1 по ГОСТ~2.301.

4.3.10 Текст каждого приложения, при необходимости, может быть разделен на разделы, подразделы, пункты, подпункты, которые нумеруют в пределах каждого приложения. Перед номером ставится обозначение этого приложения.

Приложения должны иметь общую с остальной частью документа сквозную нумерацию страниц.

4.3.11 Все приложения должны быть перечислены в содержании документа (при наличии) с указанием их номеров и заголовков.

4.3.12 Приложения, выпускаемые в виде самостоятельного документа, оформляют по общим правилам – первый лист с основной надписью по форме 2, последующие листы – по форме 2а по ГОСТ~2.104, ГОСТ~21.1101.

При необходимости такое приложение может иметь ''Содержание''.

4.3.13 Приложениям или книгам, выпущенным в виде самостоятельного документа, обозначение
присваивают как части документа с указанием в коде документа ее порядкового номера. Если приложение или книга имеют титульный лист, то на нем под наименованием документа указывают слово ''Приложение'' и его обозначение в случае двух и более приложений, например ''Приложение Б'' или ''Книга'' и ее порядковый номер, например ''Книга 6''.

4.3.14 Допускается в качестве приложения к документу использовать другие самостоятельно выпущенные конструкторские документы (габаритные чертежи, схемы и др.).

4.4 Построение таблиц

4.4.1 Таблицы применяют для лучшей наглядности и удобства сравнения показателей. Название таблицы, при его наличии, должно отражать ее содержание, быть точным, кратким. Название следует помещать над таблицей.

При переносе части таблицы на ту же или другие страницы название помещают только над первой частью таблицы.

4.4.2 Таблицы, за исключением таблиц приложений, следует нумеровать арабскими цифрами сквозной нумерацией.

Таблицы каждого приложения обозначают отдельной нумерацией арабскими цифрами с добавлением перед цифрой обозначения приложения. Если в документе одна таблица, она должна быть обозначена ''Таблица 1'' или ''Таблица В.1'', если она приведена в приложении В.

Допускается нумеровать таблицы в пределах раздела. В этом случае номер таблицы состоит из номера раздела и порядкового номера таблицы, разделённых точкой.

4.4.3 На все таблицы документа должны быть приведены ссылки в тексте документа, при ссылке следует писать слово ''таблица'' с указанием ее номера.

4.4.4 Заголовки граф и строк таблицы следует писать с прописной буквы, а подзаголовки граф – со строчной буквы, если они составляют одно предложение с заголовком, или с прописной буквы, если они имеют самостоятельное значение. В конце заголовков и подзаголовков таблиц точки не ставят. Заголовки и подзаголовки граф указывают в единственном числе.

4.4.5 Таблицы слева, справа и снизу, как правило, ограничивают линиями.

Разделять заголовки и подзаголовки боковика и граф диагональными линиями не допускается.

Горизонтальные и вертикальные линии, разграничивающие строки таблицы, допускается не проводить, если их отсутствие не затрудняет пользование таблицей.

Заголовки граф, как правило, записывают параллельно строкам таблицы. При необходимости допускается перпендикулярное расположение заголовков граф.

Головка таблицы должна быть отделена линией от остальной части таблицы.

Высота строк таблицы должна быть не менее 8~мм.

4.4.6 Таблицу, в зависимости от ее размера, помещают под текстом, в котором впервые дана ссылка на неё, или на следующей странице, а при необходимости, в приложении к документу.

Допускается помещать таблицу вдоль длинной стороны листа документа.

4.4.7 Если строки или графы таблицы выходят за формат страницы, ее делят на части, помещая одну часть под другой или рядом, при этом в каждой части таблицы повторяют ее головку и боковик. При делении таблицы на части допускается ее головку или боковик заменять соответственно номером граф и строк. При этом нумеруют арабскими цифрами графы и (или) строки первой части таблицы.

Слово ''Таблица'' указывают один раз слева над первой частью таблицы, над другими частями пишут слова ''Продолжение таблицы'' с указанием номера (обозначения) таблицы.

Если в конце страницы таблица прерывается и ее продолжение будет на следующей странице, в первой части таблицы нижнюю горизонтальную линию, ограничивающую таблицу, не проводят.

4.4.8 Графу ''Номер по порядку'' в таблицу включать не допускается. Нумерация граф таблицы арабскими цифрами допускается в тех случаях, когда в тексте документа имеются ссылки на них, при делении таблицы на части, а также при переносе части таблицы на следующую страницу в соответствии с рисунком 4.

При необходимости нумерации показателей, параметров или других данных порядковые номера следует указывать в первой графе (боковике) таблицы непосредственно перед их наименованием. Перед числовыми значениями величин и обозначением типов, марок и т.п. порядковые номера не
проставляют.

4.4.9 Если все показатели, приведённые в графах таблицы, выражены в одной и той же единице физической величины, то ее обозначение необходимо помещать над таблицей справа, а при делении таблицы на части – над каждой ее частью.

Для сокращения текста заголовков и подзаголовков граф отдельные понятия заменяют буквенными
обозначениями, установленными ГОСТ~2.321, или другими обозначениями, если они пояснены в тексте или приведены на иллюстрациях, например D – диаметр, Н – высота, L – длина.

Показатели с одним и тем же буквенным обозначением группируют последовательно в порядке возрастания индексов.

4.4.10 Ограничительные слова ''более'', ''не более'', ''менее'', ''не менее'' и др. должны быть помещены в одной строке или графе таблицы с наименованием соответствующего показателя после обозначения его единицы физической величины, если они относятся ко всей строке или графе. При этом после наименования показателя перед ограничительными словами ставится запятая.

4.4.11 Обозначение единицы физической величины, общей для всех данных в строке, следует указывать после ее наименования. Допускается при необходимости выносить в отдельную строку (графу) обозначение единицы физической величины.

4.4.12 Если в графе таблицы помещены значения одной и той же физической величины, то обозначение единицы физической величины указывают в заголовке (подзаголовке) этой графы.

4.4.17 Заменять кавычками повторяющиеся в таблице цифры, математические знаки, знаки процента и номера, обозначение марок материалов и типоразмеров изделий, обозначения нормативных документов не допускается.

4.4.18 При отсутствии отдельных данных в таблице следует ставить прочерк (тире).






























