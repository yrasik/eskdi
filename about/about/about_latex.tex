\section{Система компьютерного набора \LaTeX}
\sectionmark{Система компьютерного набора \LaTeX}

\LaTeX\ – система компьютерного набора, предназначена для создания научных и математических документов высокого типографского качества. Система также пригодна для создания других видов документов, начиная от обычных писем, и заканчивая целыми книгами.

\subsection{Краткий обзор философии и истории}

\TeX\ – это компьютерная программа, созданная Дональдом Кнутом. Она предназначена для вёрстки текста и математических формул. Кнут начал писать \TeX\ в 1977~году, чтобы исследовать потенциал цифрового печатного оснащения, которое в то время приобретало распространение в полиграфической промышленности. Он надеялся, что сможет преломить тенденцию деградации типографского качества, от чего, по его мнению, страдали его собственные книги и статьи. \TeX, в его настоящем виде, вышел в свет в 1982~году, с некоторыми усовершенствованиями, добавленными в 1989~году для лучшей поддержки восьмибитных символов и многих языков. \TeX\ известен чрезвычайной стабильностью, способностью работать на многих компьютерных платформах и операционных системах, а также практически полным отсутствием ошибок. Номер версии \TeX\ сходится к числу $\pi$ и сейчас равняется 3.14159.

\LaTeX\ - является надстройкой над системой \TeX\ в виде набора макропакетов и макрокоманд, и является системой более высокого уровня, чем \TeX. \LaTeX\ был написан Лесли Лампортом. Он использует \TeX\ как наборщика.

В 1994~году пакет \LaTeX\ был обновлён командой \LaTeX3 во главе с Франком   Миттельбахом, с целью включения некоторых давно ожидаемых улучшений, а также объединения всех усовершенствованных версий, которые появились со времени выхода версии \LaTeX~2.09 много лет назад. Чтобы не путать эту новую версию со старой, она называется \LaTeX2e. Далее идёт описание именно \LaTeX2e.

В среде \LaTeX, \LaTeX\ берет на себя роль дизайнера книги, используя \TeX\ как наборщика. Но \LaTeX\ – это ''лишь'' компьютерная программа, и потому требует чётких инструкций. Автор должен предоставить дополнительную информацию, которая описывает логическую структуру его работы – заголовки разделов, цитаты, примеры, формулы и т.д.. Эта информация записывается в текст в виде команд \LaTeX.

Это существенно отличается от подхода \textit{WYSIWYG} (What you see is what you get – что Вы видите, то и получите), принятого у большинства современных текстовых процессоров, таких как MS~Word или Corel~WordPerfect. В этих программах, авторы форматируют документ интерактивно, одновременно с набором текста на компьютере. В процессе работы они могут видеть на экране, какой вид будет иметь документ, когда он будет напечатан.

\SetEmptyPage
Используя \LaTeX\, как правило, невозможно увидеть конечную картину во время набора текста. Отформатированный документ однако можно просмотреть на экране после обработки файла \LaTeX. При необходимости, исправления можно сделать, непосредственно, перед печатью.

\LaTeX\ позволяет предотвратить ошибки форматирования, принуждая автора декларировать логическую структуру его документа. Затем \LaTeX\ выбирает наиболее подходящий макет документа.



\subsection{Достоинства и недостатки системы}

Основные преимущества \LaTeX над обычными текстовыми процессорами:
\begin{enumerate}
  \item Удобство набора математических формул.
  \item Пользователю нужно выучить лишь несколько понятных команд, которые определяют логическую структуру документа. Ему практически никогда не нужно возиться собственно с макетом документа.
  \item Легко создаются даже сложные структуры, такие, как примечания, содержание, библиография, предметный указатель, и прочее.
  \item Для решения многих типографских задач, которые не поддерживаются прямо базовым \LaTeX ом, есть свободно распространяемые дополнительные пакеты. Например, существуют пакеты для включения PostScript – графики, или для форматирования библиографии в точном соответствии с конкретными стандартами. 
  \item \LaTeX\ вынуждает авторов писать хорошо структурированные документы, так как именно так \LaTeX\ и работает – определяя структуру. 
  \item Очень сложно писать неструктурированные и неорганизованные документы.
  \item \TeX, механизм форматирования \LaTeX2e, – чрезвычайно мобильный и свободно доступный. Поэтому система работает практически на всех существующих платформах.
  \item Входными данными для \LaTeX\ является обычный текстовый файл в ASCII – формате. (документ может состоять из несколько файлов, одновременно создаваемых несколькими людьми).  
  \item Возможность использовать все преимущества систем версионирования проекта (''Subversion'' ''Mercurial'' и др...), применимые для текстовых файлов в ASCII – формате.
\end{enumerate}


Некоторые недостатки:
\begin{enumerate}
  \item \LaTeX\ плохо работает у невнимательного пользователя.
  \item Пользователь должен иметь представление о гипертекстовой разметке документа.
  \item Хотя некоторые параметры могут быть настроены в границах заранее изготовленных макетов, создать совершенно новый макет документа не очень просто, и это занимает много времени. 
  \item Несмотря на некоторые ободряющие первые шаги, Вы никогда не сможете полностью постичь суть Логической Разметки. 
\end{enumerate}














