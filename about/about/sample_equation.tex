\section[Примеры применения формул]{справочное}
\sectionmark{Примеры применения формул}

\subsection{Формула номер один}

Уравнение~(\ref{eq:n_x}) преобразования напряжения с учётом мультипликативных погрешностей блоков имеет вид:
\begin{eqnarray}
N'_x=U_xK_1(1+\delta_1)K_{\text{У1}}(1+\delta_{\text{У1}})K_{\text{У2}}(1+\delta_{\text{У2}})
\eqnewline{\times}
	K_{\text{У3}}(1+\delta_{\text{У3}})K_{\text{Д}}(1+\delta_Д)K_{\text{АЦП}}(1+\delta_{\text{АЦП}}),
\label{eq:n_x}
\end{eqnarray}
\par где $\delta_1$~– погрешность коэффициента передачи предварительного блока (делителя или повторителя);
\par $\delta_{\text{У1}}$~– погрешность коэффициента передачи усилителя У1;
\par $\delta_{\text{У2}}$~– погрешность коэффициента передачи усилителя У2;
\par $\delta_{\text{У3}}$~– погрешность коэффициента передачи усилителя У3;
\par $\delta_{\text{АЦП}}$~– погрешность коэффициента передачи АЦП;
\par $\delta_\text{Д}$~– погрешность коэффициента передачи детектора.
\\
\\


\subsection{Формула номер два}

\subsubsection{Подпункт в приложении}

Расхождение частот в процентах равно
\begin{equation}
\triangle\% = \frac{|F_{\text{оп.реал.}} - F_{\text{оп.ном.}}|}{F_{\text{оп.ном.}}} * 100 = \frac{|36,923 - 36,864|}{36,864} * 100 = 0,16.
\label{eq:n_y}
\end{equation}

\subsubsection{Ещё подпункт в приложении}