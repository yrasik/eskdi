\section{Полезные ссылки} \label{p:refs}
\sectionmark{Полезные ссылки}

\par 1. ''\verb"http://www.tug.org/texlive/"'' – Домашняя страница проекта ''\TeX~Live''.
\par 2. ''\verb"http://www.bitbucker.org/fat_angel/opengostfont/downloads"'' – Ссылка на файлы \\ шрифтов ''\verb"OpenGost Type A TT"'' и ''\verb"OpenGost Type B TT"''.
\par 3. ''\verb"http://www.tex.uniyar.ac.ru/doc.htm"'' – Ссылки на ресурсы и документацию по \LaTeX.
\par 4. ''\verb"http://gostedu.ru/"'' – Ресурс с Государственными стандартами, доступными по сети.
\par 5. ''\verb"http://www.bullzip.com/products/pdf/info.php"'' – Страница многоформатного виртуального принтера ''\verb"Bullzip PDF Printer"''. Программа то платная , то бесплатная.
\par 6. ''\verb"http://www.pdfsam.org/"'' – Страница программы  ''\verb"PDFsam"'' для редактирования *.pdf – документов. Кроссплатформенная, бесплатная.
\par 7. ''\verb"http://www.gimp.org/"'' – Страница программы для создания и переконвертации изображений ''\verb"GIMP"''. Кроссплатформенная, бесплатная, открытый исходный код.
\par 8. ''\verb"http://www.imagemagick.org/script/index.php"'' – Страница программы для переконвертации изображений ''\verb"ImageMagick"''. Кроссплатформенная, бесплатная, открытый исходный код.
\par 9. ''\verb"http://mooedit.sourceforge.net/index.html"'' – Страница программы для редактирования кода ''\verb"medit"''. Кроссплатформенная, бесплатная, открытый исходный код.
\par 10. ''\verb"http://www.xm1math.net/texmaker/"'' – Страница программы для редактирования кода и проверки орфографии ''\verb"''Texmaker''"''. Бесплатная.
\par 11. ''\verb"http://pdf-xchange-viewer.en.softonic.com/"'' – Страница программы \\''\verb"PDF-XChange Viewer"'', которая вносит пометки, комментарии в pdf. Бесплатная.


