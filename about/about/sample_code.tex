\section{Примеры включения кода в текст документа}
\sectionmark{Примеры включения кода в текст документа}

Код на языке С++ производит проверку создания дескриптора и вывода соответствующего сообщения:
\begin{lstlisting}[language=C++, style=FrameStyle, keepspaces = true]
  //проверка создания дескриптора
  if (!target_file)
  {
    *log << "ERROR: Error creating '" << FullFileName << "' file;" << endl;
    return false;
  }
  *log << "Creating '" << FullFileName << "' file;" << endl;
\end{lstlisting}

Код модуля на VerilogHDL, выделяющего фронты из медленной импульсной последовательности:
\begin{lstlisting}[language=Verilog, style=FrameStyle, inputencoding=utf8x, extendedchars=\true, keepspaces = true]
//Модуль, выделяющий фронты из медленной импульсной последовательности
module GOIR
(
  input  wire CLK,Reset,DIN,
  output wire PositiveFront,
  output wire NegativeFront
);

parameter Mem_default = 2'b0;

 reg [1:0] Mem;

 always @(posedge CLK) // Сдвиговый регистр
    if (Reset)
      Mem <= #1 Mem_default;
    else
      Mem <= #1 {Mem[0], DIN};

 assign PositiveFront = ~ Mem[1] & Mem[0];
 assign NegativeFront = ~ Mem[0] & Mem[1];

endmodule //GOIR
\end{lstlisting}


Наберите команду:
\begin{lstlisting}[language=, style=FrameStyle]
nand read 33000000 0 200000 
\end{lstlisting}
(считать 0x200000 байт из NAND–Flash–памяти начиная с адреса 0x00000000 в область памяти начиная с адреса 0x33000000). Появится отчёт:

\begin{lstlisting}[language= , style=FrameStyle]
NAND read: device 0 offset 0x0, size 0x200000
 2097152 bytes read: OK
\end{lstlisting}











