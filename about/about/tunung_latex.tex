\section{Установка и настройка окружения ''\TeX~Live~2014''}
\sectionmark{Установка и настройка окружения ''\TeX~Live~2014''}

\subsection{Сведения о шрифтах}

При установке окружения, основную проблему БОЛЬШЕ НЕ составляет установка русских шрифтов коллекции ''PSCyr''! Они не нужны! Достаточно системных шрифтов, благодаря ''xelatex''.

Пример применения шрифтов, заданных в преамбуле, приведён ниже:
\begin{itemize}
  \item  \textrm{romanfont}
  \item  \textrm{\bf romanfont bold}
  \item  \textrm{\it romanfont italic}
  \item  \texttt{monofont}
  \item  \textsf{sansfont}
\end{itemize}



\subsection{ОС ''Windows''}


\subsubsection{Установка ''\TeX~Live~2014''}

Скачайте образ диска дистрибутива ''\TeX~Live~2014'' c сайта проекта (приложение~\ref{p:refs}).

Смонтируйте диск в виртуальный привод компакт–дисков (использовав программу ''DAEMON~Tools'' или ей подобную).

Следуйте инструкциям инсталлятора (рисунок~\ref{p:texlive2010}).

\begin{figure}[H]\center
  \captionsetup{singlelinecheck=true} %центрируем подрисуночную подпись
  \includegraphics*[scale=0.65]{./about/texlive2010}
  \caption{Инсталлятор \TeX~Live~2014} \label{p:texlive2010}
\end{figure}


\subsubsection{Установка виртуального принтера ''Bullzip PDF Printer''}

Виртуальный принтер нужен для печати рисунков из разных приложений (CAD - систем). Так же он удобен, если предполагается, что в оконечном документе будут страницы, отличные от формата А4 (схемы в приложении). Эти листы иногда удобнее создавать в других приложениях, а затем распечатывать их в *.pdf –файлы с помощью данного принтера.

Скачайте дистрибутив ''Bullzip PDF Printer'' с его домашней страницы (приложение~\ref{p:refs}) и установите, следуя инструкциям инсталлятора. Может понадобиться дополнительная установка ява-машины. Возможно придётся найти раннюю версию продукта, не требующую денежных вложений.


\subsubsection{Установка программы ''PDFsam'' для редактирования *.pdf – документов}

Данная программа теперь не нужна, так как пакет умеет вставлять листы разных размеров самостоятельно в процессе сборки.

Программы ''PDFsam'' предназначена для замены листов в *.pdf – файлах. 

Скачайте дистрибутив ''PDFsam'' с его домашней страницы (приложение~\ref{p:refs}) и установите, следуя инструкциям инсталлятора.


\subsubsection{Установка программы ''GIMP'' для редактирования растровых изображений}

Скачайте дистрибутив ''GIMP'' с его домашней страницы (приложение~\ref{p:refs}) и установите, следуя инструкциям инсталлятора.


\subsubsection{Установка программы ''medit'' для редактирования кода}

''medit'' - это мультиязычный редактор кода с подсветкой синтаксиса, кроссплатформенный, с открытым исходным кодом, поддерживает широйий набор языковых кодировок.

Скачайте дистрибутив ''medit'' с его домашней страницы (приложение~\ref{p:refs}) и установите, следуя инструкциям инсталлятора.


\subsubsection{Установка программы ''Texmaker'' для редактирования кода и проверки орфографии}

''Texmaker'' - это свободный, кроссплатформенный редактор кода \TeX с подсветкой синтаксиса, быстрым подбором тегов, проверкой орфографии.

''Texmaker'' работает с кодировкой ''utf-8''. Чтобы перевести исходный \TeX - файл в кодировку ''utf-8'' нужно воспользоваться ''medit''.

Скачайте дистрибутив ''Texmaker'' с его домашней страницы (приложение~\ref{p:refs}) и установите, следуя инструкциям инсталлятора.

Для включения русской орфографии нужно отдельно скачать словари от ''OpenOffice'' (ищутся по тэгу ''словари для русского языка OpenOffice''). Скачивается файл ''dict\_ru\_RU-0.6.oxt''. Он отлично открывается архиватором, например 7-zip'ом. Файлы ''ru\_RU.dic'', ''ru\_RU.aff'' помещаются в каталог с установленной программой ( обычно в ''c:/Program Files (x86)/Texmaker/'').

Далее, нужно зайти в меню ''Texmaker'': Настройка -> Настроить Texmaker -> Редактор и в поле ''словарь для проверки орфографии'' указать файл ''ru\_RU.dic'', согласно рисунку~\ref{p:Texmaker}.

\begin{figure}[H]\center
  \captionsetup{singlelinecheck=true} %центрируем подрисуночную подпись
  \includegraphics*[scale=0.47]{./about/Texmaker}
  \caption{Настройка русской орфографии в ''Texmaker''} \label{p:Texmaker}
\end{figure}




\subsection{ОС ''Linux''}

Работа пакета не была опробована в Linux, но стоит попробовать...






