%Этот файл создан автоматически сценарием mallab.m
\section{Пример работы совместно с Matlab} \label{app:matlab}
\sectionmark{Пример работы совместно с Matlab}

Идея заключается в создании m--файла сценария таким образом, чтобы он выполнял следующие функции:
\begin{itemize}
\item выполнял необходимые расчёты;
\item строил графики и печатал их в файлы;
\item все пояснения к расчётам печатал в файл *.tex;
\item включал в этот же файл ссылки на рирсунки из Matlab;
\item печатал формулы в формате LaTeX.
\end{itemize}

Один из вариантов решения представлен здесь.
Формула~(\ref{matlab_eq:1}) создана оперетором Matlab latex():
\begin{eqnarray}
y=x^2
\label{matlab_eq:1}
\end{eqnarray}
Это может быть иногда полезно.
\newpage

\begin{figure}[!h]\center
\captionsetup{singlelinecheck=true}
\includegraphics*[scale=0.4]{./about/matlab/matlab_figure_001}
\caption{Графики функций} \label{matlab_figure_001}
\end{figure}
\newpage
Простые формулы $y=x^3$ также можно вставлять в текст.

\begin{figure}[!h]\center
\captionsetup{singlelinecheck=true}
\includegraphics*[scale=0.4]{./about/matlab/matlab_figure_002}
\caption{Графики функций} \label{matlab_figure_002}
\end{figure}

Ниже приведён листинг данного m--файла
\inputminted[fontsize=\small, linenos, breaklines, numbersep=2mm, xleftmargin=5mm]{matlab}{./about/matlab/mallab.m}



%\fvset{frame=lines, numbers=left, numbersep=5pt, fontshape=sl, fontsize=\footnotesize}
%\VerbatimInput{./about/matlab/mallab.m}

