\section{Пример построения диаграммы состояний}
\sectionmark{Пример построения диаграммы состояний}

\begin{figure}[!h]\center
  \captionsetup{singlelinecheck=true} 

\tikzset{
    state/.style={
           rectangle,
           rounded corners,
           draw=black, very thick,
           minimum height=2em,
           inner sep=2pt,
           text centered,
           },
}

\begin{tikzpicture}[->,>=stealth']
 % Position of QUERY 
 % Use previously defined 'state' as layout (see above)
 % use tabular for content to get columns/rows
 % parbox to limit width of the listing
 \node[state] (QUERY) 
 {\begin{tabular}{l}
  \textbf{Query}\\
  \parbox{4cm}{\begin{itemize}
   \item Start
   \item Parameter $Q$
   \item Zufallszahl aus \mbox{$[0, 2^Q-1]$} in Slotzhler $SC$
  \end{itemize}
  }\\[4em]
  \textbf{QueryAdjust}\\
  \parbox{4cm}{\begin{itemize}
   \item Variiere Q
   \item neue Zufallszahl
  \end{itemize}
  }
 \end{tabular}};
  
 % State: ACK with different content
 \node[state,    	% layout (defined above)
  text width=3cm, 	% max text width
  yshift=2cm, 		% move 2cm in y
  right of=QUERY, 	% Position is to the right of QUERY
  node distance=6.5cm, 	% distance to QUERY
  anchor=center] (ACK) 	% posistion relative to the center of the 'box'
 {%
 \begin{tabular}{l} 	% content
  \textbf{Ack}\\
  \parbox{2.8cm}{Besttigen mit $RN_{16}$}
 \end{tabular}
 };
 
 % STATE QUERYREP
 \node[state,
  below of=ACK,
  yshift=-2cm,
  anchor=center,
  text width=3cm] (QUERYREP) 
 {%
 \begin{tabular}{l}
  \textbf{QueryRep}\\
  \parbox{2.8cm}{Dekrementiere Slotzhler}
 \end{tabular}
 };

 % STATE EPC
 \node[state,
  right of=ACK,
  node distance=5cm,
  anchor=center] (EPC) 
 {%
 \begin{tabular}{l}
  Antwort: \textbf{EPC}\\
  \parbox{4cm}{Mit nchstem \mbox{\textbf{QueryRep}} als ''inventoried'' markieren.}
 \end{tabular}
 };

 % draw the paths and and print some Text below/above the graph
 \path (QUERY) 	edge[bend left=20]  node[anchor=south,above]{$SC_n=0$}
                                    node[anchor=north,below]{$RN_{16}$} (ACK)
 (QUERY)     	edge[bend right=20] node[anchor=south,above]{$SC_n\neq 0$} (QUERYREP)
 (ACK)       	edge                                                     (EPC)
 (EPC)       	edge[bend left]                                          (QUERYREP)
 (QUERYREP)  	edge[loop below]    node[anchor=north,below]{$SC_n\neq 0$} (QUERYREP)
 (QUERYREP)  	edge                node[anchor=left,right]{$SC_n = 0$} (ACK);

\end{tikzpicture}

\caption{Диаграмма состояний системы} \label{p:state_diagr}
\end{figure}



